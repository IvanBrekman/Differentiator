\documentclass{article}
\usepackage[utf8]{inputenc}

\usepackage[T2A]{fontenc}
\usepackage[russian]{babel}
\usepackage{amssymb}
\usepackage{geometry}
\geometry{verbose,a4paper,tmargin=2cm,bmargin=2cm,lmargin=2.0cm,rmargin=2.0cm}

\title{Differentiator}
\author{IvanBrekman}
\date{01.12.2021 23:30:15, Wednesday}

\begin{document}

\maketitle

Продифференуируем функцию f = $ \frac{-3 \cdot ({x}^{2} + 4 \cdot x + -5)}{(5 - {x}^{(x \cdot (3 - {x}^{2}))})} $\newline
Не сложно заметить, что \newline
($ x $)$ ^\prime $ = $ 1 $\newline
\newline
Легко привести к виду \newline
($ x $)$ ^\prime $ = $ 1 $\newline
\newline
Не сложно заметить, что \newline
($ {x}^{2} $)$ ^\prime $ = $ 2 \cdot {x}^{(2 - 1)} \cdot 1 $ = $ 2 \cdot x $\newline
\newline
Очевидно, что \newline
($ 3 $)$ ^\prime $ = $ 0 $\newline
\newline
Не трудно заметить, что \newline
($ 3 - {x}^{2} $)$ ^\prime $ = $ 0 - 2 \cdot x $ = $ -1 \cdot 2 \cdot x $\newline
\newline
Не требует объяснений тот факт, что \newline
($ x $)$ ^\prime $ = $ 1 $\newline
\newline
Преобразуя получаем \newline
($ x \cdot (3 - {x}^{2}) $)$ ^\prime $ = $ 1 \cdot (3 - {x}^{2}) + x \cdot -1 \cdot 2 \cdot x $ = $ 3 - {x}^{2} + x \cdot -1 \cdot 2 \cdot x $\newline
\newline
Не сложно заметить, что \newline
($ {x}^{(x \cdot (3 - {x}^{2}))} $)$ ^\prime $ = $ {x}^{(x \cdot (3 - {x}^{2}))} \cdot ((3 - {x}^{2} + x \cdot -1 \cdot 2 \cdot x) \cdot \ln{(x)} + \frac{1}{x} \cdot x \cdot (3 - {x}^{2})) $\newline
\newline
Обратим внимание, что \newline
($ 5 $)$ ^\prime $ = $ 0 $\newline
\newline
Не трудно заметить, что \newline
($ 5 - {x}^{(x \cdot (3 - {x}^{2}))} $)$ ^\prime $ = $ 0 - {x}^{(x \cdot (3 - {x}^{2}))} \cdot ((3 - {x}^{2} + x \cdot -1 \cdot 2 \cdot x) \cdot \ln{(x)} + \frac{1}{x} \cdot x \cdot (3 - {x}^{2})) $ = $ -1 \cdot {x}^{(x \cdot (3 - {x}^{2}))} \cdot ((3 - {x}^{2} + x \cdot -1 \cdot 2 \cdot x) \cdot \ln{(x)} + \frac{1}{x} \cdot x \cdot (3 - {x}^{2})) $\newline
\newline
Если данный кусок вызывавает сомнения, то обратитесь к доказательтсву великой теоремы Ферма \newline
($ -5 $)$ ^\prime $ = $ 0 $\newline
\newline
По известным фактам \newline
($ x $)$ ^\prime $ = $ 1 $\newline
\newline
Никто не будет ШоКиРоВаН, если мы скажем, что \newline
($ 4 $)$ ^\prime $ = $ 0 $\newline
\newline
Обратим внимание, что \newline
($ 4 \cdot x $)$ ^\prime $ = $ 0 \cdot x + 4 \cdot 1 $ = $ 4 $\newline
\newline
По известным фактам \newline
($ 4 \cdot x + -5 $)$ ^\prime $ = $ 4 + 0 $ = $ 4 $\newline
\newline
Не требует объяснений тот факт, что \newline
($ x $)$ ^\prime $ = $ 1 $\newline
\newline
Можно увидеть, что \newline
($ {x}^{2} $)$ ^\prime $ = $ 2 \cdot {x}^{(2 - 1)} \cdot 1 $ = $ 2 \cdot x $\newline
\newline
Легко привести к виду \newline
($ {x}^{2} + 4 \cdot x + -5 $)$ ^\prime $ = $ 2 \cdot x + 4 $\newline
\newline
Очевидно, что \newline
($ -3 $)$ ^\prime $ = $ 0 $\newline
\newline
Преобразуя получаем \newline
($ -3 \cdot ({x}^{2} + 4 \cdot x + -5) $)$ ^\prime $ = $ 0 \cdot ({x}^{2} + 4 \cdot x + -5) + -3 \cdot (2 \cdot x + 4) $ = $ -3 \cdot (2 \cdot x + 4) $\newline
\newline
Легко привести к виду \newline
($ \frac{-3 \cdot ({x}^{2} + 4 \cdot x + -5)}{(5 - {x}^{(x \cdot (3 - {x}^{2}))})} $)$ ^\prime $ = $ \frac{(-3 \cdot (2 \cdot x + 4) \cdot (5 - {x}^{(x \cdot (3 - {x}^{2}))}) - -3 \cdot ({x}^{2} + 4 \cdot x + -5) \cdot -1 \cdot {x}^{(x \cdot (3 - {x}^{2}))} \cdot ((3 - {x}^{2} + x \cdot -1 \cdot 2 \cdot x) \cdot \ln{(x)} + \frac{1}{x} \cdot x \cdot (3 - {x}^{2})))}{{(5 - {x}^{(x \cdot (3 - {x}^{2}))})}^{2}} $\newline
\newline
Таким образом имеем f$^\prime$ = $ \frac{(-3 \cdot (2 \cdot x + 4) \cdot (5 - {x}^{(x \cdot (3 - {x}^{2}))}) - -3 \cdot ({x}^{2} + 4 \cdot x + -5) \cdot -1 \cdot {x}^{(x \cdot (3 - {x}^{2}))} \cdot ((3 - {x}^{2} + x \cdot -1 \cdot 2 \cdot x) \cdot \ln{(x)} + \frac{1}{x} \cdot x \cdot (3 - {x}^{2})))}{{(5 - {x}^{(x \cdot (3 - {x}^{2}))})}^{2}} $\newline

\end{document}
