\documentclass{article}
\usepackage[utf8]{inputenc}

\usepackage[russian]{babel}
\usepackage{amssymb}
\usepackage{geometry}
\geometry{verbose,a4paper,tmargin=2cm,bmargin=2cm,lmargin=2.0cm,rmargin=2.0cm}

\title{Differentiator}
\author{IvanBrekman}
\date{05.12.2021 13:23:23, Sunday}

\begin{document}

\maketitle

\section {Производная}
Продифференуируем функцию f = $ 3 \cdot \cos{(x)} + {x}^{5} $\newline
Несложно заметить, что \newline
($ x $)$ ^\prime $ = $ 1 $\newline
\newline
Легко привести к виду \newline
($ {x}^{5} $)$ ^\prime $ = $ 5 \cdot {x}^{(5 - 1)} \cdot 1 $ = $ 5 \cdot {x}^{4} $\newline
\newline
Несложно заметить, что \newline
($ x $)$ ^\prime $ = $ 1 $\newline
\newline
Очевидно, что \newline
($ \cos{(x)} $)$ ^\prime $ = $ -1 \cdot \sin{(x)} \cdot 1 $ = $ -1 \cdot \sin{(x)} $\newline
\newline
Нетрудно заметить, что \newline
($ 3 $)$ ^\prime $ = $ 0 $\newline
\newline
Не требует объяснений тот факт, что \newline
($ 3 \cdot \cos{(x)} $)$ ^\prime $ = $ 0 \cdot \cos{(x)} + 3 \cdot -1 \cdot \sin{(x)} $ = $ 3 \cdot -1 \cdot \sin{(x)} $\newline
\newline
Преобразуя получаем \newline
($ 3 \cdot \cos{(x)} + {x}^{5} $)$ ^\prime $ = $ 3 \cdot -1 \cdot \sin{(x)} + 5 \cdot {x}^{4} $\newline
\newline
Таким образом имеем f$^\prime$ = $ 3 \cdot -1 \cdot \sin{(x)} + 5 \cdot {x}^{4} $
\end{document}
